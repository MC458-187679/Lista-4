\documentclass[a4paper, 14pt]{extarticle}

\usepackage[brazilian]{babel}
% \usepackage[utf8]{inputenc}
\usepackage[T1]{fontenc}
\usepackage[margin=1.5cm,top=1.8cm,noheadfoot=true]{geometry}

\usepackage{float, pgf, caption, subcaption}

% \input{secoes}
% \input{teorema}
\makeatletter

\usepackage{amsthm, amsmath, amssymb, bm, mathtools}
\usepackage{enumitem, etoolbox, xpatch}
% \usepackage[mathcal]{euscript}
% \usepackage[scr]{rsfso}
\usepackage{mathptmx}
\usepackage{relsize, centernot, tikz, xcolor}


%%%% QED symbols %%%%
\def\qed@open{\ensuremath{\square}}
\def\qed@open@small{\ensuremath{\mathsmaller\qed@open}}

\def\qed@fill{\ensuremath{\blacksquare}}
\def\qed@fill@small{\ensuremath{\mathsmaller\qed@fill}}

\definecolor{qed@gray}{gray}{0.8}
\def\qed@gray{\ensuremath{\color{qed@gray}\blacksquare}}
\def\qed@gray@small{\ensuremath{\color{qed@gray}\mathsmaller\blacksquare}}

\def\qed@both@cmd#1#2{\begin{tikzpicture}[baseline=#2]
    \draw (0,0) [fill=qed@gray] rectangle (#1,#1);
\end{tikzpicture}}
\def\qed@both{\qed@both@cmd{0.6em}{0.2ex}}
\def\qed@both@small{\qed@both@cmd{1ex}{0ex}}

\def\showAllQED{
    \qed@open ~ \qed@open@small \\
    \qed@fill ~ \qed@fill@small \\
    \qed@gray ~ \qed@gray@small \\
    \qed@both ~ \qed@both@small
}

%% escoha do QED %%
\renewcommand{\qedsymbol}{\qed@fill@small}

% marcadores de prova
\newcommand{\direto}[1][~]{\ensuremath{(\rightarrow)}#1}
\newcommand{\inverso}[1][~]{\ensuremath{(\leftarrow)}#1}

% fontes
% conjunto potencia
\DeclareSymbolFont{boondox}{U}{BOONDOX-cal}{m}{n}
\DeclareMathSymbol{\pow}{\mathalpha}{boondox}{"50}

% somatorio
\DeclareSymbolFont{matext}{OMX}{cmex}{m}{n}
\DeclareMathSymbol{\sum@d}{\mathop}{matext}{"58}
\DeclareMathSymbol{\sum@t}{\mathop}{matext}{"50}
\undef\sum
\DeclareMathOperator*{\sum}{\mathchoice{\sum@d}{\sum@t}{\sum@t}{\sum@t}}
\DeclareMathOperator*{\bigsum}{\mathlarger{\mathlarger{\sum@d}}}

% phi computer modern
\DeclareMathAlphabet{\gk@mf}{OT1}{cmr}{m}{n}
\let\old@Phi\Phi
\def\Phi{\gk@mf{\old@Phi}}

% união elem por elem
\DeclareMathOperator{\wcup}{\mathaccent\cdot\cup}

% familia de conjuntos
\undef\fam
\DeclareMathAlphabet{\fam}{OMS}{cmsy}{m}{n}

% alguns símbolos
\undef\natural
\DeclareMathOperator{\real}{\mathbb{R}}
\DeclareMathOperator{\natural}{\mathbb{N}}
\DeclareMathOperator{\integer}{\mathbb{Z}}
\DeclareMathOperator{\complex}{\mathbb{C}}
\DeclareMathOperator{\rational}{\mathbb{Q}}
\def\symdif{\mathrel{\triangle}}
\def\midd{\;\middle|\;}
% \def\pow{\mathcal{P}}

% operações com mais espaçamento
\def\cupp{\mathbin{\,\cup\,}}
\def\capp{\mathbin{\,\cap\,}}

% alguns operadores
\DeclareMathOperator{\Dom}{Dom}
\DeclareMathOperator{\Img}{Im}

% marcadores de operadores
\def\inv{^{-1}}
\def\rel#1{\use@invr{\!\mathrel{#1}\!}}
\def\nrel#1{\use@invr{\!\centernot{#1}\!}}
\def\dmod#1{\ (\mathrm{mod}\ #1)}
\def\cgc#1{{\textnormal{[}#1\textnormal{]}}}
\def\cgp#1{{\textnormal{(}#1\textnormal{)}}}

% marcador com inverso reduzido
\def\use@invr#1{%
    \begingroup%
        \edef\inv{\inv\!}%
        #1%
    \endgroup%
}

% delimiters
\def\abs#1{{\lvert\,#1\,\rvert}}
% \DeclarePairedDelimiter{\abs}{\lvert}{\:\rvert}

\makeatother


% math display skip
\newcommand{\reducemathskip}[1][0.5em]{%
    \setlength{\abovedisplayskip}{1pt}%
    \setlength{\belowdisplayskip}{#1}%
    \setlength{\abovedisplayshortskip}{#1}%
    \setlength{\belowdisplayshortskip}{#1}%
}

% url linking problems
\def\url#1{\href{#1}{\texttt{#1}}}
% vermelho
\def\red#1{\textcolor{red}{#1}}

\def\lmref#1{\thmref[lema ]{#1}}

\usepackage{xparse, caption, booktabs}
\usepackage[hidelinks]{hyperref}
\usepackage[nameinlink, brazilian]{cleveref}
\crefformat{equation}{#2eq.~#1#3}
\crefformat{definition}{#2def.~#1#3}
\crefformat{proof}{#2dem.~#1#3}
\usepackage[section, newfloat]{minted}
\definecolor{sepia}{RGB}{252,246,226}
\setminted{
    bgcolor = sepia,
    % style   = pastie,
    frame   = leftline,
    autogobble,
    samepage,
    python3,
}
\setmintedinline{
    bgcolor={}
}

\theoremstyle{plain}
\newtheorem*{hypothesis}{Hipótese}
\newtheorem*{hypothesisf}{Hipótese Fortalecida}

\newtheoremstyle{definicao}% name of the style to be used
  {}% measure of space to leave above the theorem. E.g.: 3pt
  {}% measure of space to leave below the theorem. E.g.: 3pt
  {}% name of font to use in the body of the theorem
  {}% measure of space to indent
  {\bf}% name of head font
  {:}% punctuation between head and body
  {.8em}% space after theorem head; " " = normal interword space
  {\thmnote{\textbf{#3}}}% Manually specify head
\theoremstyle{definicao}
\newtheorem*{definition}{Definição}

\NewDocumentCommand{\seq}{ s m O{n} O{\in\natural} }
    {\IfBooleanTF{#1}
        {\ensuremath{\left({#2}_{#3}\right)}}
        {\ensuremath{\left({#2}_{#3}\right)_{{#3}{#4}}}}}


\usepackage{titling, titlesec, enumitem}
% \usepackage{algorithmic}
\usepackage{clrscode3e, xspace}
\title{\vspace{-2.5cm}\Large Lista de Exercícios Avaliativa 4 \\ \normalsize MC458 - 2s2020 - Tiago de Paula Alves - 187679}
\preauthor{}\author{}\postauthor{}
\predate{}\date{}\postdate{}
\posttitle{\par\end{center}\vskip-1em}

\titleformat{\section}{\large\bfseries}{\thesection}{.8em}{}
\titlespacing*{\section}{0pt}{.5em plus .2em minus .2em}{.5em plus .2em}

\newlist{casos}{enumerate}{2}
\setlist[casos]{wide,labelwidth={\parindent},listparindent={\parindent},parsep={\parskip},topsep={0pt},label=\textbf{Caso \arabic*}:}
\setlist[casos,2]{label=\textbf{Caso \arabic{casosi}\alph*}:}

\newlist{ncasos}{description}{2}
\setlist[ncasos]{wide,listparindent={\parindent},parsep={\parskip},topsep={0pt}}

\titleformat{\section}[runin]
    {\titlerule{}\vspace{1ex}\normalfont\Large\bfseries}{}{1em}{}[.]
\titleformat{\subsection}[runin]
    {\normalfont\large\bfseries}{}{1em}{}[)]

% linha final da página ou seção
\newcommand{\docline}[1][\\]{%
    #1\noindent\rule{\textwidth}{0.4pt}%
    \pagebreak%
}
\newcommand{\itemdsep}{
    \noindent\hfil\rule{0.5\textwidth}{.2pt}\hfil
    \vskip1em
}


\usepackage{tikz}
\usetikzlibrary{calc,trees,positioning,arrows,fit,shapes,calc}

\DeclareMathSymbol{\mlq}{\mathord}{operators}{``}
\DeclareMathSymbol{\mrq}{\mathord}{operators}{`'}

\usepackage{fancyhdr}
\pagestyle{empty}

% \usepackage{showframe}
\begin{document}

    \maketitle
    \thispagestyle{empty}

    \noindent\rule{\textwidth}{0.4pt}
    \begin{center}\Large\vskip-0.5em
        CORRIGIR A QUESTÃO \textbf{2}.
    \end{center}

    \section{1}
    \begingroup
        \begin{codebox}
    \Procname{$\proc{Vizinhança-I-Ésimo}(V, n, i, l)$}
    \li \Comment Acha o $i$-ésimo menor elemento de $V$.
    \li $\proc{Select-BFPRT}(V, 1, n, i)$ \label{linha:acha1}
    \li $\id{val} \gets V[i]$   \label{linha:acha2}
    \li
    \li \Comment Monta os pares com a distância (absoluto
    \li \Comment da diferença) de cada elemento até o
    \li \Comment $i$-ésimo, junto com o próprio elemento
    \li Seja $P[1 \twodots n]$ um novo vetor de pares de inteiros. \label{linha:criap}
    \li \kw{para} $j = 1$ \kw{até} $n$ \label{linha:forp}
        \Do
    \li     $\id{dist} \gets \big|V[j] - \id{val}\big|$ \label{linha:dist}
    \li     $P[j] \gets \big(\id{dist}\bm{,} V[j]\big)$ \label{linha:par}
        \End
    \li $\proc{Select-BFPRT}(P, 1, n, l)$ \label{linha:separa}
    \li
    \li \Comment Copia o resultado para um novo vetor.
    \li Seja $A[1 \twodots l] $ um novo vetor de inteiros. \label{linha:criaa}
    \li \kw{para} $j = 1$ \kw{até} $l$ \label{linha:fora}
        \Do
    \li     $\big(\id{dist}\bm{,} \id{num}\big) \gets P[j]$ \label{linha:numa}
    \li     $A[j] \gets \id{num}$ \label{linha:montaa}
        \End
    \li
    \li \kw{retorna} $A$ \label{linha:reta}
\end{codebox}

\itemdsep{}

O algoritmo acima se baseia no fato de que o $\proc{Select-BFPRT}$, como apresentado em aula, além de encontrar o $i$-ésimo menor elemento, também particiona o vetor $V$ de forma que elementos menores que o $i$-ésimo se encontram nas primeiras $i-1$ posições e os maiores nas últimas $n-i$ posições. Isso implica que o $i$-ésimo elemento vai estar na posição $V_i$.

\itemdsep{}

\renewcommand{\qedsymbol}{}
\begin{proof}[\textbf{Corretude}]
    O primeiro passo do algoritmo é encontrar o $i$-ésimo menor elemento de $V$, que após o $\proc{Select-BFPRT}$ se encontra na posição $i$ de $V$ (linhas \ref{linha:acha1} e \ref{linha:acha2}). Em seguida, é montado um novo vetor $P$ de pares, onde $P_j = \left(\text{dist}(V_j, V_i), V_j\right)$ e $\text{dist}(a, b) = \abs{a - b}$ (linhas \ref{linha:criap} à \ref{linha:par}). Os pares $P_j$ são comparados em ordem lexicográfica, em que o primeiro elemento do par tem maior precedência na comparação.

    Após isso, $\proc{Select-BFPRT}$ é chamado com $P$ para encontrar o $l$-ésimo menor, de acordo com a distância do valor até $V_i$ (linha \ref{linha:separa}). Devido ao algoritmo de particionamento, temos que todos os pares $P_j$, com $j \leq l$, compõe as $l$ menores distâncias até o $i$-ésimo elemento, ou seja, temos ali os $l$ elementos mais próximos do $i$-ésimo.

    A última etapa (linhas \ref{linha:criaa} a \ref{linha:montaa}) apenas copia os resultados encontrados anteriormente, descartando as distâncias utilizadas para comparação e particionamento.
\end{proof}

\itemdsep{}

\begin{proof}[\textbf{Complexidade}]
    Como o algoritmo $\proc{Select-BFPRT}$ é $O(n)$, então as linhas $\ref{linha:acha1}$ até $\ref{linha:separa}$ executam em tempo também $O(n)$. Além disso, as linhas \ref{linha:criaa} a \ref{linha:montaa} executam em $O(l)$. Portanto, o algoritmo como um todo tem complexidade $O(n + l)$.

    No entanto, é importante notar que $l \leq n$. Logo, podemos afirmar que o algoritmo executa em tempo $O(n)$, como requerido.
\end{proof}

    \endgroup

    \docline[]

    \section{2}
    \begingroup
        Chorãozinho está correto, a divisão em $\left\lfloor\frac{n}{3}\right\rfloor$ grupos é menos eficiente que usando $\left\lfloor\frac{n}{5}\right\rfloor$ grupos no algoritmo BFPRT, para $n$ suficientemente grande.

\itemdsep{}

Dividindo um vetor $A$ de tamanho $n$ em grupos de 3, a recuperação das medianas dos grupos terá complexidade $\Theta(\left\lceil\frac{n}{3}\right\rceil) = \Theta(n)$. A partir disso, a mediana das medianas será encontrada em tempo $T(\left\lceil\frac{n}{3}\right\rceil)$ e, após o particionamento em $O(n)$, estará na posição $k$. Para a análise de pior caso, podemos assumir que $k$ não é a solução e a próxima etapa da busca será na maior da partições, com tamanho $n_p = \max\{k - 1, n - k\}$ e levando tempo $T(n_p)$.

Note que existem pelo menos $\left\lceil \frac{1}{2} \left\lceil \frac{n}{3} \right\rceil - 1 \right\rceil$ grupos com medianas menores que o elemento em $k$, desconsiderando o grupo com menos de 3 elementos, que pode ser o próprio grupo de $k$. Em cada um desses, teremos garantidos 2 elementos menores que $A_k$.

Podemos considerar, então, um dos piores casos para a complexidade, em que $k$ assume seu menor valor possível. Assim:
\begin{align*}
    \#\left(A_<\right)
    &\leq 2 \left\lceil \frac{1}{2} \left\lceil \frac{n}{3} \right\rceil - 1 \right\rceil + 1 \\
    &\leq 2 \left(\frac{1}{2} \left(\frac{n}{3} + 1\right) - 1 + 1\right) + 1\\
    &\leq \frac{n}{3} + 2
\end{align*}

Nesse caso, a maior das partições terá $n_p = n - k \geq \frac{2n}{3} - 2 \geq \left\lfloor\frac{2n}{3}\right\rfloor - 2$ elementos. A análise considerando o maior valor possível de $k$ chega em uma mesma cota inferior para $n_p$.

Portanto, considerando que $T(n)$ é crescente e $a$ positivo, a recorrência do tempo de execução do algoritmo no pior caso será:
\begin{align*}
    T(n)
    &= T(\lceil n / 3 \rceil) + T(n_p) + \Theta(n) \\
    &\geq T\left(\left\lceil \frac{n}{3} \right\rceil\right) + T\left(\left\lfloor\frac{2n}{3}\right\rfloor - 2\right) + a n && \text{(quando $n > 4$)}
\end{align*}

Assumindo $0 < T(1) \leq T(2) \leq t_0$, podemos mostrar que $T(n) \geq c n \lg n$, para algum $c > 0$. Por fim, temos então que, quando o vetor é divido em grupos de 3 elementos, $T(n) \in \Omega(n \lg n)$, que é assintoticamente menos eficiente que o algoritmo original.

\itemdsep{}

\begin{proof}[\textbf{Demonstração}]
    Considere a constante $c = \frac{a}{5}$.

    \begin{ncasos}
        \item[Caso base:] Seja $3 \leq n \leq 8$. Logo,
        \[
            T(n) \geq T(\lceil n / 3 \rceil) + T(n_p) + a n \geq a n
        \]
        Mas,
        \[
            c n \lg n \leq \frac{a}{5} n \lg 8 = \frac{3}{5} a n \leq T(n)
        \]

        \item[Hipótese indutiva:] Suponha que $T(k) \geq c k \lg k$ para todo $3 \leq k < n$ e algum $n > 8$.

        \item[Passo indutivo:] Seja $n \geq 9$. Logo,
        \begin{align*}
            T(n)
            &\geq T\left(\left\lceil \frac{n}{3} \right\rceil\right) + T\left(\left\lfloor\frac{2n}{3}\right\rfloor - 2\right) + a n \\
            &\geq \left[ c \left\lceil \frac{n}{3} \right\rceil \lg\left\lceil \frac{n}{3} \right\rceil \right] + \left[ c \left(\left\lfloor\frac{2n}{3}\right\rfloor - 2\right) \lg\left(\left\lfloor\frac{2n}{3}\right\rfloor - 2 \right) \right] + a n \\
            &\geq c \frac{n}{3} \lg\left(\frac{n}{3}\right) + c \left(\frac{2n}{3} - 3\right) \lg\left(\frac{2n}{3} - 3 \right) + a n \\
            &= c \frac{n}{3} \lg\left(\frac{n}{3}\right) + c \left(\frac{2n}{3} - 3\right) \lg\left(\frac{2n - 9}{3}\right) + a n \\
            &\geq c \frac{n}{3} \lg\left(\frac{n}{3}\right) + c \left(\frac{2n}{3} - 3\right) \lg\left(\frac{n}{3}\right) + a n \\
            &= c \frac{3n}{3} \lg\left(\frac{n}{3}\right) - 3 c \lg\left(\frac{n}{3}\right) + a n \\
            &= c n \lg n  - c n \lg 3 - 3 c \lg n + 3 c \lg 3 + an \\
            &> c n \lg n  - c n \lg 3 - 3 c n + an \\
            &= c n \lg n + (a - (3 + \lg 3) c) n \\
            &\geq c n \lg n + (a - 5 c) n \\
            &= c n \lg n
        \end{align*}
    \end{ncasos}
\end{proof}

    \endgroup

    \docline[]

    \section{3}
    \begingroup
        A única tarefa possível é a de Chorãozinho, ou seja, é impossível que um algoritmo para este problema tenha complexidade de pior caso $O(n)$.

\itemdsep{}

Note que um algoritmo que resolve o problema dado não tem acesso à capacidade $c(R)$ de um resistor $R$, podendo apenas saber se $c(R) < c(r)$, $c(R) = c(r)$ ou $c(R) > c(r)$, para um resistor $r$ de cor diferente. Não é possível nem mesmo saber $c(R) - c(r)$ ou $c(R) / c(r)$, que poderia servir como uma medida de distância entre os resistores. Portanto, o algoritmo deve ser baseado puramente em comparações das capacidades.

Supondo então um algoritmo que resolve o problema, ele deve retornar uma permutação dos resistores amarelos $A$ e outra dos verdes $V$, em pares $(A_i, V_j)$ com $0 < i, j \leq n$. Logo, esse algoritmo deve ser capaz de escolher entre $\left(n!\right)^2$ soluções, das quais $n!$ são corretas, já que a posição dos pares não é importante, bastando que $c(A_i) = c(V_j)$ em cada par $(A_i, V_j)$ da solução.

Podemos ver o algoritmo como uma Árvore de Decisão Ternária, onde cada nó interno é uma comparação de capacidade entre um $A_i$ e um $V_j$, as arestas são os resultados das camparações e as folhas são as soluções. Portanto, o algoritmo, cuja árvore tem altura $h$, deve conter $3^h \geq \left(n!\right)^2$ folhas. Logo, $h \geq \log_3 \left(\left(n!\right)^2\right) = 2 \log_3 \left(n!\right)$.

Apesar disso, um algoritmo ótimo poderia organizar as folhas de forma que as $n!$ soluções corretas apareçam em uma mesma subárvore, decidindo entre elas de forma costante. Nesse caso, a subárvore teria altura $h' = \left\lceil\log_3(n!)\right\rceil$, e a solução poderia ser encontrada na altura $h^* = h - h' \geq 2 \log_3(n!) - \left\lceil\log_3(n!)\right\rceil > \log_3(n!) - 1$, ou seja, $h^* \geq \log_3(n!)$. Entretanto, isso não reduziria a complexidade do algoritmo.

Note que $(n!)^2 \geq n^n$, portanto:
\begin{align*}
    h^* &\geq \log_3(n!) \\
    &\geq \frac{1}{2} \log_3\left(n^n\right) \\
    &= \frac{1}{2} \frac{n \lg n}{\lg 3} \\
    &= \frac{n \lg n}{2 \lg 3} \\
    &\geq \frac{n \lg n}{4}
\end{align*}

Com isso, podemos afirmar que o tempo de execução do algoritmo em pior caso será no mínimo $T(n) \geq h^* \geq \frac{1}{4} n \lg n$. Por fim, temos então que $T(n) \in \Omega(n \lg n)$, ou seja, $T(n) \not\in O(n)$.

Um algoritmo também poderia ser implementado com base em comparações binárias, representado por uma Árvore de Decisão Binária. Isso não alteraria a complexidade, mas poderia aumentar o número de comparações necessárias.

    \endgroup

    \docline[]

\end{document}
